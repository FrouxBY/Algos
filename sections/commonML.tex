%!TEX root = ../master.tex
\chapter{Common Machine Learning algorithms}\label{ch:introduction}

This chapter is dedicated to the most common ML algorithms, a major part of the notes come from the mml-books.com \todo{Add bibtex reference}
\todo[inline]{Find better paragraph layout}

\section{Linear Regression}

	The Linear regression problem correspond to find a linear mapping $f(x)$ based on noisy observation $y = f(x) + \epsilon$, where $\epsilon$ is a noise. 
	Finding the regression function require : 
	\begin{itemize}
		\item Choice of parameters (function classes, dimension)
		\item Choice of probabilistic model (Loss function, ...)
		\item Avoiding under and overfitting
		\item Modeling uncertainty on data
	\end{itemize}

	\begin{definition}
			\begin{itemize}
				\item $p(x, y)$ is te joint distribution
				\item $p(x)$ and $p(y)$ are the marginal distributions
				\item $p(y|x)$ is the conditional distribution of $y$ given $x$
			 	\item in the context of regression, $p(y|x)$ is called likelihood, $p(x|y)$ the posterior, $p(x)$ the prior and $p(y)$ the marginal likelihood or evidence.
			 	\item they are related by the Bayes' Theorem : $p(x|y) = \frac{p(y|x)p(x)}{p(y)}$ 
			\end{itemize}
		\end{definition}

	\subsection{Conjugacy} % (fold)
	\label{sub:conjugacy}
		In order to compute the posterior, we require some calculations which imply the prior, and can be intractable as a closed form. But given a likelihood, it can exist prior which give closed-form solution for the posterior. This is the principle of conjugacy (between the likelihood and the prior)
	% subsection conjugacy (end)
	\subsection{Maximum Likelihood Estimation (MLE)}

		\subsubsection*{Closed-Form Solution}

			In some cases, a closed-form solution exist, which make computation easy (but not necesseraly cheap)
	\subsubsection{Maximum A Posteriori Estimation (MAP)}

\section{Gradient Descent}

	\subsection{Simple Gradient Descent}

	\subsection{Gradient Descent with Momentum}

	\subsection{Stochastic Gradient Descent}


\section{Model Selection and Validation}

	\subsection{Cross-Validation}

	\subsection{Marginal Likelihood}

\section{Bayesian Linear Regression}

	\subsection{Mean and Variance}

	\subsection{Sample function}

