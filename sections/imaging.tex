%!TEX root = ../master.tex
\chapter{Imaging} % (fold)
\label{cha:imaging}

\section{Image Classification}

	\subsection{Features and Classifier}

	\subsection{Feature Extraction - without ML}

	\begin{itemize}
	 	\item Why features ?
	 	\item What features should be : \\
	 		– Distinctive and discriminative
			– Local (to enable establishing correspondences)
			– Invariant to viewpoint changes or transformations (translations/rotations)
			– Invariant to illumination changes
			– Efficient to compute
			– Robust to noise and blurring
			– Should be hierarchical

		\item Many options possible : – Intensities
									– Gradient
									– Histogram
									– SIFT
									– SURF
									– BRIEF
		\item Pixel level bad : Not discriminative, Localiser and does not represent local pattern, not invariant to transformation. 
		\item Patch Level : More discriminative, Not necessarily rotation invariant, semi-localised. Use Convolution for patch. 
		\item Image Level : discriminative, Not localised, not necessarily rotation invariant. Get a feature map where each pixel correspond to local pattern
	 \end{itemize} 

	 How to make rotation invariant ? $ \rightarrow $ Use Histogram.
	 \subsubsection{Filtering}

		- Use filter to identify features such as gradient, edge detection.\\
		- Gradient are invariant to absolute illumination.

	\subsubsection{SIFT}

	\begin{itemize}
		\item Scale-space Extrema Detection : For all scale, compute convolution with gaussian (blur at different level) and compute the difference of each scaled guassian, then Detect local extrema of Differences
		\item Keypoint Localisation : 
		\item Orientation Assignment : Take a small window around the keypoint location.An orientation histogram with 36 bins
		covering 360 degrees is created.
		Each pixel votes for an orientation bin,
		weighted by the gradient magnitude
		after applying a Gaussian filter with
		he keypoint scale !).
		The keypoint will be assigned an
		orientation, which is the mode of the
		distribution
		\item Keypont descriptors.
	\end{itemize}

	\subsubsection{OHoG (Histogram of Gradient)} % (fold)
	\label{sub:hog_}
	
	% subsection hog (end)
	
	\subsubsection{SURF (Speeded-Up Robust Features)} % (fold)
	\label{sub:surf}
	
	% subsection surf (end)

	\subsubsection{BRIEF (Binary Robust Independant Elementary Features)} % (fold)
	\label{sub:brief_}
	
	% subsection brief_ (end)

	\subsubsection{Local binary Patterns} % (fold)
	\label{sub:local_binary_patterns}
	
	% subsection local_binary_patterns (end)

	\subsubsection{Haar features} % (fold)
	\label{sub:haar_features}
	
	Convolution ineficient $\rightarrow$ Faster computation via integral images S
	% subsection haar_features (end)
	
	\subsection{with ML} % (fold)
	\label{sub:with_ml}
	
		\todo{Add Classical ML pipeline for computer vision (images, features, classification)}

		Classification or regression model, many options : Logistic regression, Naives Bayes, K-Nearest neighbors, Support vector Machines (does not deal good with computer vision \todo{why ?}, Boosting, Decision/Random forsest, Neural network.

		Boosting and Decision trees are Ensemble forests. 
	% subsection with_ml (end)

	\todo[inline]{add bias-variance error explanation in common ML}

% chapter imaging (end)