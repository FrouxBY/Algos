%!TEX root = ../master.tex
\chapter{Argumentation Framework} % (fold)
\label{cha:argumentation_framework}

This chapter are notes from the Imperial Course Machine Arguing from Francesca Toni. \todo{add ref}

\paragraph{introduction} % (fold)
\label{par:introduction}

	Argument Framework are a field in AI which provide way of evaluate any debate problem. It is useful to resolve conflict, to explain decision or to deal with incomplete information. 
% paragraph introduction (end)
\section{Abstract Argumentation}

	\subsection{Simple AA}

		\begin{definition}
		 		an  \textbf{AA framework} is a set $\mathrm{Args}$ of arguments and a binary relation $\mathrm{attacks}$. $(\alpha, \beta)\in \mathrm{attacks}$ means $\alpha$ attacks $\beta$.
		\end{definition} 

		\paragraph{Semantics in AA} 
			In order to define a "winning" set of argument, we need to provide semantics over the the framework. This is like recipes which determine good set of arguments. 

		\begin{definition}
			\begin{itemize}
				\item \textbf{conflict-free}
				\item \textbf{admissible:} c-f and attacks each attacking argument.
				\item \textbf{preferred:} maximally admissible.
				\item \textbf{complete:} admissible + contains each argument it defends.
				\item \textbf{stable:} c-f + attacks each argument not in it.
				\item \textbf{grounded:} minimally complete.
				\item \textbf{sceptically preferred:} Intersection of all prefered.
				\item \textbf{ideal:} maximal admissible and contain in all prefered (i.e. in the sceptically preferred).
			\end{itemize}
		\end{definition}

		\begin{definition}
			\textbf{Semi-stable extension:} complete such as $A\cup A^+$ is maximal. $A^+$ is the set of attacked argument by $A$.
		\end{definition}

		\todo{add ref to ASPARTIX and CONARG}

	\subsection{Algorithms for AA}

		\paragraph*{Computing Grounded extensions}
		Use the same algorithms as grounded labelling, but only output the IN arguments as the grounded extensions.\\
		the grounded extensions in unique. 

		\paragraph{Computing the grounded labelling:}
		Here is an algorithm to compute a grounded labelling\\

		\begin{algorithm}[H]
		\KwData{An AA Framework}
		\KwResult{The grounded Labelling}
		Label all unatacked argument with IN \;
		\While{The IN and the OUT are not stable}
		{
			Label OUT the arguments attacks by IN \;
			Label IN the arguments only attacked by OUT\;
		}
		Label the still unlabelled UNDEC\;
		\caption{Computing the grounded labelling}
		\end{algorithm}

		\paragraph{Computing membership in preferred/grounded/ideal extensions;} In order to compute membership, we use Dispute Tree.

		We compute a dispute tree for an argument, and apply the different semantics which are easier to compute on a tree than on a graph.
		\todo{add algos of computing dispute tree + def of semantics}

		\paragraph{Computing stable extension:} We use answer set programming with logical program. 
		

	\subsection{AA with Suppport}

		\subsubsection[BAA]{Bipolar Abstract Argumentation}
			We add a \textbf{Support} relation to a classic AA Framework (BAA = $\langle Args, Attacks, Supports \rangle$). \\There are different semantics :

			\paragraph*{Semantics in BAA with deductive support}
			We can deduce an AA Framework from a BAA with $Attacks' = Attacks \cup Attacks_{sup} \cup Attacks_{s-med}$ with 
			\begin{definition}
				\begin{itemize}
					\item $Attacks_{sup}$ is \textbf{the supported attacks}  $\implies \alpha$ attacks every argument that its supports attacks (supports of supports are supports)
					\item $Attacks_{s-med}$ is the super mediated attacks  $ \implies \alpha$ attacks every argument whose supports an argument attacked or $attacked_{sup}$  by $\alpha$ 
				\end{itemize}
			\end{definition}

			Then, we apply the AA semantics to $\langle Args, Attacks' \rangle$. Those semantics are the d-X semantics, where X replace every semantic from AA (grounded, complete, etc.)
			\paragraph*{QuAD Framework}
			We focus here one the QuAD (\textbf{Quantitative Argument Debate}) which add a numerical strenght to any argument, and give rule for updating strenght regarding the supporters or attackers.
			

			\begin{definition}
				\begin{itemize}
					\item We define the QuAD Framework as $\langle \mathcal{A}, \mathcal{C}, \mathcal{P}, \mathcal{R}, \mathcal{BS}\rangle $ Where $\mathcal{A}$ is the set of answer arguments, $\mathcal{C}$ is the set of con args, $\mathcal{P}$ are the pro args, which are disjoint, $\mathcal{R}$ the relation (with $\mathcal{R}^+$ the supporters and $\mathcal{R}^-$ the attackers) and $\mathcal{BS}$ the Base Score, which give score to each args. 
					\item The \textbf{Base funcion} is $f(v_1, v_2) = v_1 + (1-v_1)v_2$
					\item the \textbf{aggregation function} as a recursive definition : 
					\begin{align*}
						& n=0 \implies \mathcal{F}(S)=0\\
						& n=1 \implies \mathcal{F}(S)= v_1\\
						& n=2 \implies \mathcal{F}(S)= f(v_1, v_2)\\
						& n>2 \implies \mathcal{F}(S)= f(\mathcal{F}(S\backslash \{v_n\}, v_n))
					\end{align*}
					\item the \textbf{Combination function} is defined by: \\
					$c(v_0, v_a, v_s) = v_0 - v_0 \cdot |v_s-v_a|$ if $v_a \geq v_s$\\
					$c(v_0, v_a, v_s) = v_0 + (1 - v_0 ) \cdot |v_s-v_a|$ if $v_a < v_s$
					\item the \textbf{Strength Function} is defined as : $\mathcal{S}(\alpha) = c(\mathcal{BS}(\alpha), \mathcal{F}(SEQ_S(\mathcal{R}^-(\alpha))), \mathcal{F}(SEQ_S(\mathcal{R}^+(\alpha))))$
				\end{itemize}
			\end{definition}

			The DF-QuAD (for \textbf{Discontinuity free}-QuAD) algorithm is simply to apply these semantics to a BAA framework, in setting initial weights to any arguments. 


	\subsection{Argument Mining}
	\subsection{AA with Preference}
		When taking into account preference in a AA, (in the form $X < Y = X$ is less prefered to $Y$) we can still use the classic AA semantics but we can improved it by using the preferences informations :
		\begin{itemize}
			\item By \textbf{deleting attacks} (an $(x, y)$ attacks succeed iff $x \nless y$) 
			\item By \textbf{reversing attacks} (if $(x, y)$ and $x < y$ then we reverse by replacing this attack by $(y, x)$).
			\item By \textbf{Selecting Amongst extension}, using the preference to select extension with the same extension (ex : differentiate 2 stable extensions). Look at \textbf{Rich PAF} for some formal definition.
		\end{itemize}
	\subsection{AA with Probabilities}

\section{Assumption-Based Argumentation}
	
	\textbf{Assumption Based Argumentation} introduce arguments as Assumptions $a \in \mathcal{A}$, or deduction $\sigma$ with premises and conclusions achevied with Rules $R \subseteq \mathcal{R}$(deductive rules, as $q \wedge a \implies p$) defined in a language $\mathcal{L}$ 
	\subsection{Simple ABA}

		We can define \textbf{attacks} in ABA : an argument $A_1 \vdash \sigma_1$ attacks $A_2 \vdash \sigma_2$ iff $\sigma_1$ is the contrary of one of the assumptions in $A_2$.
	\subsection{ABA more DDs}
	\subsection{p-acyclic ABA}

		\begin{definition}
			a \textbf{positive-acyclic ABA} is a AB framework where the dependancy graph of AB is acyclic.\\
			The \textbf{dependancy graph} is a graph of all the rules, where assumption ($\mathcal{A}$) are deleted. 
		\end{definition}

\section{ArgGame}
% chapter argumentation_framework (end)