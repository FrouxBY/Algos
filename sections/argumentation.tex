%!TEX root = ../master.tex
\chapter{Argumentation Framework} % (fold)
\label{cha:argumentation_framework}

This chapter are notes from the Imperial Course Machine Arguing from Francesca Toni. \todo{add ref}

\paragraph{introduction} % (fold)
\label{par:introduction}

	Argument Framework are a field in AI which provide way of evaluate any debate problem. It is useful to resolve conflict, to explain decision or to deal with incomplete information. 
% paragraph introduction (end)
\section{Abstract Argumentation}

	\subsection{Simple AA}

		\begin{definition}
		 		an  \textbf{AA framework} is a set $\mathrm{Args}$ of arguments and a binary relation $\mathrm{attacks}$. $(\alpha, \beta)\in \mathrm{attacks}$ means $\alpha$ attacks $\beta$.
		\end{definition} 

		\paragraph{Semantics in AA} 
			In order to define a "winning" set of argument, we need to provide semantics over the the framework. This is like recipes which determine good set of arguments. 

		\begin{definition}
			\begin{itemize}
				\item \textbf{conflict-free}
				\item \textbf{admissible:} c-f and attacks each attacking argument.
				\item \textbf{preferred:} maximally admissible.
				\item \textbf{complete:} admissible + contains each argument it defends.
				\item \textbf{stable:} c-f + attacks each argument not in it.
				\item \textbf{grounded:} minimally complete.
				\item \textbf{sceptically preferred:} Intersection of all prefered.
				\item \textbf{ideal:} maximal admissible and containing all prefered.
			\end{itemize}
		\end{definition}

		\begin{definition}
			\textbf{Semi-stable extension:} complete such as $A\cup A^+$ is maximal. $A^+$ is the set of attacked argument by $A$.
		\end{definition}

		\todo{add ref to ASPARTIX and CONARG}

	\subsection{Algorithms for AA}

		\paragraph*{Computing Grounded extensions}
		Use the same algorithms as grounded labelling, but only output the IN arguments as the grounded extensions.\\
		the grounded extensions in unique. 

		\paragraph{Computing the grounded labelling:}
		Here is an algorithm to compute a grounded labelling\\

		\begin{algorithm}[H]
		\KwData{An AA Framework}
		\KwResult{The grounded Labelling}
		Label all unatacked argument with IN \;
		\While{The IN and the OUT are not stable}
		{
			Label OUT the arguments attacks by IN \;
			Label IN the arguments only attacked by OUT\;
		}
		Label the still unlabelled UNDEC\;
		\caption{Computing the grounded labelling}
		\end{algorithm}

		\paragraph{Computing membership in preferred/grounded/ideal extensions;} In order to compute membership, we use Dispute Tree.

		We compute a dispute tree for an argument, and apply the different semantics which are easier to compute on a tree than on a graph.
		\todo{add algos of computing dispute tree + def of semantics}

		\paragraph{Computing stable extension:} We use answer set programming with logical program. 
		

	\subsection{AA with Suppport}

		\subsubsection[BAA]{Bipolar Abstract Argumentation}
			We add a \textbf{Support} relation to a classic AA Framework (BAA = $\langle Args, Attacks, Supports \rangle$). \\There are different semantics :

			\paragraph*{Semantics in BAA with deductive support}
			We can deduce an AA Framework from a BAA with $Attacks' = Attacks \cup Attacks_{sup} \cup Attacks_{s-med}$ with 
			\begin{definition}
				\begin{itemize}
					\item $Attacks_{sup}$ is \textbf{the supported attacks}  $\implies \alpha$ attacks every argument that its supports attacks (supports of supports are supports)
					\item $Attacks_{s-med}$ is the super mediated attacks  $ \implies \alpha$ attacks every argument whose supports an argument attacked or $attacked_{sup}$  by $\alpha$ 
				\end{itemize}
			\end{definition}

			Then, we apply the AA semantics to $\langle Args, Attacks' \rangle$. Those semantics are the d-X semantics, where X replace every semantic from AA (grounded, complete, etc.)
			\paragraph*{QuAD}
			We focus here one the QuAD (\textbf{Quantitative Argument Debate}) which add a numerical strenght to any argument, and give rule for updating strenght regarding the supporters or attackers.

			\todo{Add DF-QuAD rules and algorithm}
	\subsection{Argument Mining}
	\subsection{AA with Preference Probabilist}

\section{Assumption-Based Argumentation}

	\subsection{Simple ABA}
	\subsection{ABA more DDs}
	\subsection{p-acyclic ABA}

\section{ArgGame}
% chapter argumentation_framework (end)